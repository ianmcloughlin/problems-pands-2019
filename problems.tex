\documentclass[a4paper, 12pt]{exam}

  % Enables the use of colour.
  \usepackage{xcolor}
  % Syntax high-lighting for code. Requires Python's pygments.
  \usepackage{minted}
  % Enables the use of umlauts and other accents.
  \usepackage[utf8]{inputenc}
  % Diagrams.
  \usepackage{tikz}
  % Settings for captions, such as sideways captions.
  \usepackage{caption}
  % Symbols for units, like degrees and ohms.
  \usepackage{gensymb}
  % Latin modern fonts - better looking than the defaults.
  \usepackage{lmodern}
  % Allows for columns spanning multiple rows in tables.
  \usepackage{multirow}
  % Better looking tables, including nicer borders.
  \usepackage{booktabs}
  % More math symbols.
  \usepackage{amssymb}
  % More math fonts, like \mathbb.
  \usepackage{amsfonts}
  % More math layouts, equation arrays, etc.
  \usepackage{amsmath}
  % More theorem environments.
  \usepackage{amsthm}
  % More column formats for tables.
  \usepackage{array}
  % Adjust the sizes of box environments.
  \usepackage{adjustbox}
  % Better looking single quotes in verbatim and minted environments.
  \usepackage{upquote}
  % Better blank space decisions.
  \usepackage{xspace}
  % Better looking tikz trees.
  \usepackage{forest}
  % URLs.
  \usepackage[hidelinks]{hyperref}
  % Plotting.
  \usepackage{pgfplots}
  % Calculates the number of pages.
  \usepackage{lastpage}
  % Styling the abstract.
  \usepackage{abstract}
  
  % Various tikz libraries.
  % For drawing mind maps.
  \usetikzlibrary{mindmap}
  % For adding shadows.
  \usetikzlibrary{shadows}
  % Extra arrows tips.
  \usetikzlibrary{arrows.meta}
  % Old arrows.
  \usetikzlibrary{arrows}
  % Automata.
  \usetikzlibrary{automata}
  % For more positioning options.
  \usetikzlibrary{positioning}
  % Creating chains of nodes on a line.
  \usetikzlibrary{chains}
  % Fitting node to contain set of coordinates.
  \usetikzlibrary{fit}
  % Extra shapes for drawing.
  \usetikzlibrary{shapes}
  % For markings on paths.
  \usetikzlibrary{decorations.markings}
  % For advanced calculations.
  \usetikzlibrary{calc}
  
  % GMIT colours.
  \definecolor{gmitblue}{RGB}{20,134,225}
  \definecolor{gmitred}{RGB}{220,20,60}
  \definecolor{gmitgrey}{RGB}{67,67,67}

  % Set minted up.
  \usemintedstyle{manni}
  \setminted{baselinestretch=1.2, bgcolor=gmitgrey!10}


  %%%%%% CHANGE
  % Name of the module.
  \newcommand{\modulename}{Programming and Scripting}
  % Year of delivery.
  \newcommand{\projectyear}{2019}
  % Title of project.
  \newcommand{\projectname}{Problem Set \projectyear}
  % Due date.
  \newcommand{\duedate}{last commit on or before March 31\textsuperscript{st}}
  %%%%%%

  % Change the displayed name of the references section.
  \renewcommand{\refname}{\selectfont\large References} 

  % Change the headers and footers.
  \pagestyle{headandfoot}
  \header{\textbf{\projectname}}{}{\modulename}
  \footer{}{Page \thepage\ of \numpages}{}
  
  % Use a blank cover page.
  \begin{coverpages}
  \end{coverpages}

  % Set up the title.
  \title{\projectname}
  \author{\modulename}
  \date{Due: \duedate}

\begin{document}
  
\maketitle

\noindent
This document contains the instructions for \projectname{} for \modulename{}.
The ten problems below should be completed by you during the semester.
The lecturer will indicate which problems you should solve on different weeks.
Your solutions will be assessed towards the end of the semester and will be worth 50\% of your marks for this module.

The marking scheme is given below, and please note that one of the categories is for a pragmatic attitude to work, part of which involves working on the exercises as indicated by the lecturer during the weeks of the semester.
It is not expected that you get every program right first time.
So long as an attempt is made when indicated by the lecturer, this will count as a good approach.
It is important that you keep working on any incomplete problems until the deadline.

Please note that all students are bound by the Quality Assurance Framework~\cite{gmitqaf} at GMIT which includes the Code of Student Conduct and the Policy on Plagiarism.
The onus is on the student to ensure they do not, even inadvertently, break the rules.
A clean and comprehensive git~\cite{git} history (see below) is the best way to demonstrate that your submission is your own work.
It is, however, expected that you draw on works that are not your own and you should systematically reference those works to enhance your submission.

\subsection*{How to submit}
During the semester you will learn how to use the version control software git~\cite{git} to track your work.
You will sync your work with GitHub~\cite{github} (you may use another provider if you wish) and provide the URL to your GitHub repository using the submission form on the Moodle page.
You should consult the videos on the Moodle page for information on how to do this.

I suggest you consider making your repository publicly available (the default) so that prospective employers may view it.
Should you wish to, you may make it private so long as you make it available to the lecturer.
You can submit your URL at any time as git will track what work was done before the deadline and what work was completed afterwards.
In general, only work completed before the deadline will count.
However, in borderline cases or exceptional circumstances work done after the deadline may be considered, with or without penalty.

\subsection*{Marking scheme}
  This problem set will be worth 50\% of your marks for this module.
  The following marking scheme will be used to mark your submission out of 100\%.
  The examiner's overall impression of the assignment may influence marks in each individual component.

  \begin{center}
    \begin{tabular}{llp{8.4cm}}
      \toprule
      25\% & \textbf{Research} & Investigation of each problem and its solution, as evidenced by clean and well-commented code. \\
      \midrule
      25\% & \textbf{Development} & Clear, well-written, and efficient code. \\
      \midrule
      25\% & \textbf{Consistency} & Good planning and pragmatic attitude to work as evidenced by commit history. \\
      \midrule
      25\% & \textbf{Documentation} & Concise descriptions of solutions in comments and README. \\
      \bottomrule
    \end{tabular}
  \end{center}

  \subsection*{Advice for students}
  \begin{itemize}
    \item
      You must be able to explain your assignment during and after its completion.
      If you had trouble understanding something in the first place, you will likely have trouble explaining it a couple of weeks later.
      Write a short explanation of it into your submission, so that you can jog your memory later.
    \item
      Everyone is susceptible to procrastination and disorganisation.
      You are expected to be aware of this and take reasonable measures to avoid them.
    \item
      Students have problems from time to time.
      Some of these are unavoidable, such as acute family issues or illness.
      In such cases allowances regarding assignments can sometimes be made.
      Students should be able to show that up until an issue arose they had completed a reasonable and proportionate amount of work and took reasonable steps to avoid preventable issues.
    \item
      Go easy on yourself --- this is one assignment in one module.
      It will not define you or your life.
      A higher overall course mark should not be determined by a single assignment.
  \end{itemize}


  \bibliographystyle{plain}
  \bibliography{bibliography}


\subsection*{Questions}
  
  For some some of the following questions, examples of running the program are given in grey.
  Lines beginning with a dollar sign are typed at the command line, and the other lines are part of the programs.

  \begin{questions}

    \question
    Write a program that asks the user to input any positive integer and outputs the sum of all numbers between one and that number.
    \begin{minted}{bash}
$ python sumupto.py
Please enter a positive integer: 10
55
    \end{minted}

    \question
      Write a program that outputs whether or not today is a day that begins with the letter T.
      An example of running this program on a Thursday is as follows.
    \begin{minted}{bash}
$ python begins-with-t.py
Yes - today begins with a T.
    \end{minted}
      An example of running it on a Wednesday is as follows.
      \begin{minted}{bash}
$ python begins-with-t.py
No - today does not begin with a T.
      \end{minted}


      \question
      Write a program that prints all numbers between 1,000 and 10,000 that are divisible by 6 but not 12.
      \begin{minted}{bash}
$ python divisors.py
1002
1014
1026
etc
9990
      \end{minted}

    \question
      Write a program that asks the user to input any positive integer and outputs the successive values of the following calculation.
      At each step calculate the next value by taking the current value and, if it is even, divide it by two, but if it is odd, multiply it by three and add one.
      Have the program end if the current value is one.
      \begin{minted}{bash}
$ python collatz.py
Please enter a positive integer: 10
10 5 16 8 4 2 1
      \end{minted}

    \newpage


    \question
    Write a program that asks the user to input a positive integer and tells the user whether or not the number is a prime.
    \begin{minted}{bash}
$ python primes.py
Please enter a positive integer: 19
That is a prime.
    \end{minted}
    
    \question
    Write a program that takes a user input string and outputs every second word.
    \begin{minted}{bash}
$ python secondstring.py
Please enter a sentence: The quick brown fox jumps over the lazy dog.
The brown jumps the dog
    \end{minted}

    \question
    Write a program that that takes a positive floating point number as input and outputs an approximation of its square root.
    \begin{minted}{bash}
$ python squareroot.py
Please enter a positive number: 14.5
The square root of 14.5 is approx. 3.8.
    \end{minted}

    \question
    Write a program that outputs today's date and time in the format ``Monday, January 10th 2019 at 1:15pm''.
    \begin{minted}{bash}
$ python datetime.py
Monday, January 10th 2019 at 1:15pm
    \end{minted}

    \question
    Write a program that reads in a text file and outputs every second line.
    The program should take the filename from an argument on the command line.
    \begin{minted}{bash}
$ python second.py moby-dick.txt
Title:  Moby Dick; or The Whale
CHAPTER 1
Call me Ishmael.  Some years ago--never mind how long
particular to interest me on shore, I thought I would sail about a
...
    \end{minted}

    \question
    Write a program that displays a plot of the functions $x$, $x^2$ and $2^x$ in the range $[0, 4]$.

  \end{questions}

\end{document}
